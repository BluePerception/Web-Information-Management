\documentclass[10pt,                    % corpo del font principale
               a4paper,                 % carta A4
               twoside,                 % impagina per fronte-retro
               openright,               % inizio capitoli a destra
               english,                 
               italian,                 
]{article}

\usepackage[utf8]{inputenc}
\usepackage[T1]{fontenc}
\usepackage[italian]{babel}
\usepackage{lmodern}

\usepackage{amsmath}
\usepackage{amsfonts}

\begin{document}

\paragraph{Domande sul motore di ricerca} \mbox{}\\
\begin{itemize}
\item \textbf{Cos'è lo spam index?} \\
Lo spamdex è l'insieme di attività svolte con lo scopo di far sì che un
sito sia indicizzato dal motore di ricerca e occupi le posizioni più alte 
della SERP (Search Engine Results Page).
Lo spam index è conosciuto anche come SEO (Search Engine Optimization) o SEP
(Search Engine Positioning). Quando si effettua una ricerca, ad ogni pagina
viene assegnato un punteggio, punteggio calcolato in base a due componenti: 
quella testuale (data dal TFIDF) e da quella ipertestuale. Il TFIDF(Term 
Frequency Inverse Document frequency) è il risultato del prodotto del TF per 
il IDF. Per calcolare il TF, si calcolala percentuale di occorenza di una 
parola in una pagina, per esempio se ci troviamo davanti a una pagina web 
formata da 100 parole e ricerchiamo la parola "pippo" che ricorre 5 volte, 
avremo un TF pari al 5\%. Risulta ovvio che articoli e preposizioni avranno 
un TF maggiore rispetto ad altre parole poichè sono di collegamento e non di 
contenuto.
L'IDF,invece, calcola l'inverso della frequenza della parola nell'insieme dei
nostri documenti in scala logaritmica. Quindi, per esempio, su 1000 pagine
l'articolo "il" appare in 980 pagine e risulta un TF=98\% mentre il
IDF=log(1/0,98\%)=0,008. Invece, se su 1000 parole la parola "pippo" ha il
10\% di frequenza il suo IDF sarà uguale a 1. Risulta chiaro come più rara è
la parola e più è alto il bonus.
Dando una definizione semplificata: un motore di ricerca prende tutte le 
pagine contententi la parola cercata e calcola di ognuna il TFIDF, se si 
ricercano più parole, si sommano i TFIDF di tutte le parole ricercate. \\

\item \textbf{Tecniche per aumentare il posizionamento} \\
Per aumentare il posizionamento, si deve cercare di aumentare il TFIDF delle 
parole chiave del sito. Per fare ciò, esistono varie tecniche sul dove 
posizionare queste keywords:
\begin{itemize}
\item \textbf{Body spam}: si inseriscono le parole nel body della pagina
HTML, è efficiente ma svantaggia la lettura per l'utente;
\item \textbf{Title spam}: si inseriscono le parole chiave nei titoli della 
pagina (tag title), il contenuto viene modificato molto meno;
\item \textbf{Meta tag spam}: le keywords vengono inserite in appositi tag 
html (i meta, appunto) che descrivono a keyword la pagina. Il vantaggio 
principale è che il contenuto non viene modificato, ma i motori di ricerca 
odierni li ignorano per l'intenso sfruttamento.
\item \textbf{Anchor text spam}: le parole chiavi vengono inserite nel testo 
dei link e il punteggio assegnato rispetto alle comuni parole è maggiore, 
anche la pagina di destinazione riceve dei benefici in termini di punteggio 
poichè viene aumentato.
\item \textbf{URL spam}: le parole chiavi si trovano nell'indirizzo della 
pagina stessa e anche'essi ricevono dei punti bonus rispetto a comuni parole 
nel testo (come per l'anchor text spam).  
\end{itemize}

\item \textbf{Quali tecniche vengono utilizzate per inserire le keywords} \\
Le parole chiavi posso venire inserite in diversi modi, per esempio:
\begin{itemize}
\item \textbf{Repetition}: la keyword viene ripetuta stando attenti al suo 
TFIDF. Questa tecnica è facile da individuare dai motori di ricerca che 
possono penalizzare il punteggio della pagina.
\item \textbf{Dumping}: vengono inseriti moltissimi termini poco usati anche 
se questi non sono inerenti al testo,risulta efficace contro quelle queries  
che includono termini relativamente rari e poco chiari: per queste queries è 
probabile che solo un paio di pagine siano rilevanti, così anche una pagina  
spam con basso livello di rilevanza/importanza potrà apparire tra i risultati 
migliori.
\item \textbf{Weaving}: inserisco nel mio sito pezzi di altri siti web in cui 
sono inserite le nostre keywords in modo random, creando un contenuto più 
interessante e valutato positivamente dai motori di ricerca che assegnaranno 
al sito un punteggio più alto in questo modo.
\item \textbf{Stitching}: viene fatto il past\&copy da siti differenti e si 
assembla il tutto ottenendo un contenuto rilevante. Più argomenti diversi 
danno inoltre un bonus globale e i SE fanno fatica a capire se il contenuto è 
copiato.
\item \textbf{Broadening}: inseriscono non solo le keywords, ma anche i 
sinonimi, le parole simili e frasi correlate. Non solo per coprire le 
richieste degli utenti, ma anche per sfruttare la tecnica di somiglia dei 
motori di ricerca così da ottenere bonus aggiuntivi. 
\end{itemize}
 
\item \textbf{Spiegare il problema e le soluzioni al term spamming} \\
Usare lo spamdex per l'utente non è molto gradevole e crea disapprovazione.
Per ovviare a questo problema, vengono utilizzate varie tecniche:
\begin{itemize}
\item \textbf{Content hiding}: il contenuto potenziante è nascosto all'utente 
e solo i motori di ricerca lo vedono.
\item \textbf{Redirection}: detta anche tecnica 302, consiste nel potenziare 
una pagina che sarà costituita con soli fini di posizionamento che indirizza
alla vera pagina. Così gli utenti vedranno la vera pagina studiata per loro 
mentre i SE si fermeranno a leggere la prima. Questa tecnica funziona solo 
con javascript poichè i SE riconoscono il meta che reindirizza mentre 
utilizzando js il meta verrà aggiunto dinamicamente e i SE non si 
accorgeranno di questo in quanto non eseguono il codice. C'è da dire che i SE 
odierni riconoscono questa tecnica, quindi per ingannarli è sufficiente 
offuscare un po' il codice.
\item \textbf{Cloaking}: è una specie di redirection che controlla se la 
pagina è richiesta da un utente o da un motore di ricerca e a seconda della
risposta ricevuta viene presentata una pagina apposita per il bot e una per 
l'utente. È una tecnica difficile da scoprire e per questo se il motore di 
ricerca scopre che si sta utilizzando questa tecnica, si viene bannati per 
un periodo.
\end{itemize}

\item \textbf{Cos'è il pagerank} \\
Per calcolare il punteggio di una pagina, viene valutato il punteggio 
relativo al pagerank, ovvero l'algoritmo assegna un peso numerico ad ogni 
elemento di un collegamento ipertestuale presente nel sito. ???
DA FINIRE CHE NON CAPISCO UN CAZZO

\end{itemize}

\paragraph{Web semantico} \mbox{}\\
\begin{itemize}
\item \textbf{Scrivere cos'è, pro e contro dell'RDF} \\
Il modello RDF(Resource Description Framework) è un framework comune che 
descrive metadati, relazioni e concetti interoperabili, ovvero permette lo
scambio di informazioni senza necessità di traduzioni in linguaggi. Si basa 
su una grammatica di base semplice: soggetto-predicato-complemento oggetto.
Questi tre elementi possono contenere due diversi tipi di dati: URI oppure 
stringhe letterali, che compongono la struttura linguistica e logica più 
semplice. L'RDF può essere rappresentato come un grafo e può essere 
rappresentato in due differenti modi: attraverso XML utilizzandone la 
struttura per alzare la complessità e arricchire il significato oppure 
attraverso le N-TRIPLE utilizzando le triplette "soggetto-predicato-comp. 
oggetto". Quest'ultimo è molto più semplice dato che è stato appositamente
fatto per RDF. Per integrare RDF con il web, si è utilizzato XHTML2 con cui 
viene introdotto RDFa che aggiunge gli attributi about(il soggetto) e 
property(il verbo) mentre il contenuto è il complemento oggetto.
Usando RDF, dove quindi le informazioni vengono rappresentate utilizzando 
i grafi, funziona molto bene l'aggregazione: unendo due grafi di conoscenza 
si genera un grafo creato tramite collegamenti automatici dei nomi del 
web(URI). Grazie a questa proprietà, RDF risulta migliore rispetto ai 
dialetti XML. Funzionano bene anche la reificazione, ovvero si riduce ad un
oggetto una asserzione, ottenendo una diversificazione di livelli e 
garantendo la monotonicità, cioè se ci si fida dell'insieme allora sono 
giusti anche i singoli pezzi che lo contengono; i contenitori, concetti 
logici di AND e OR; le variabili, oggetti logici non specificati; e la ]
monotonicità, come detto precedentemente, preso un grafo, se l'informazione
espresso in esso è vera allora tutti i sotto-grafi sono veri.
Inoltre, più struttura permette una maggiore integrità e una maggiore 
deduzione: se ho una classe che contiene un oggetto specifico, si può dedurre
che questo oggetto specifico è di tale classe anche se non è esplicitamente 
scritto.

\item \textbf{RDF-schema} \\
L'RDF-schema è lo standard che permette il supporto di ontologie, per dare 
informazioni più semplici vado quindi a fare una classificazione diminuendo 
così la complessità. La struttura informativa degli oggetti è data da: class, 
subClassOf e individual(concetto di individuo), mentre per i verbi abbiamo a 
disposizione property(la relazione, il verbo), subPropertyOf, Domain(tipo di 
relazione che delinea il verbo), range(dove la proprietà è applicata).
Grazie a questi strumenti possiamo dare informazione e categorizzarla 
\end{itemize}

\end{document}